\usepackage{fontspec}
\usepackage{sourcesanspro}
\usepackage[T1]{fontenc}

% TODO: albatian
\usefonttheme{serif}
\usefonttheme{professionalfonts}
\usepackage{mathtools}
\usepackage[tracking=true]{microtype}


\usepackage{bold-extra}
\usepackage{realscripts}
\usepackage{amsmath,bm,amssymb}
\usepackage{amsthm}
\usepackage{bbm}
\usepackage{booktabs}
\usepackage{tikz}
\usepackage[group-digits=integer,group-minimum-digits=4,group-separator={,},detect-all]{siunitx}
\usepackage[noend]{algpseudocode}
\usepackage{textpos}

\setmonofont{DejaVu Sans Mono}[Scale=MatchLowercase]
\usepackage{newunicodechar}
\newfontface{\calligraphic}{Latin Modern Math}[Scale=0.85]
\newunicodechar{𝒪}{{\normalfont\calligraphic 𝒪}}
\newunicodechar{ℬ}{{\normalfont\calligraphic ℬ}}
\newunicodechar{𝒜}{{\normalfont\calligraphic 𝒜}}
\newunicodechar{𝒟}{{\normalfont\calligraphic 𝒟}}
\newunicodechar{𝒮}{{\normalfont\calligraphic 𝒮}}
\newunicodechar{𝔼}{{\normalfont\calligraphic 𝔼}}
\newunicodechar{⋮}{{\normalfont ⋮}}
\newunicodechar{φ}{ϕ} % switched
\newunicodechar{ϕ}{φ} % switched
\newunicodechar{𝐰}{$\mathbf{w}$}
\newunicodechar{𝐯}{$\mathbf{v}$}
\newunicodechar{𝐕}{$\mathbf{V}$}
\newunicodechar{𝐡}{$\mathbf{h}$}
\newunicodechar{𝐠}{$\mathbf{g}$}
\newunicodechar{𝚺}{{\normalfont\calligraphic 𝚺}}
\newunicodechar{𝕀}{$\mathbb{I}$}
\newunicodechar{ℯ}{{\normalfont\calligraphic ℯ}}

\usepackage[style=verbose,backend=biber]{biblatex}
\addbibresource{\jobname.bib}
\setbeamerfont{footnote}{size=\tiny}
\setbeamertemplate{bibliography item}{}% Remove reference icon.
\renewcommand*{\bibfont}{\footnotesize}

\usepackage{vector} % local
\usepackage[pgfplots]{../juliaplots.sty/juliaplots}
\setpythontexpygopt{style=algforopt}
\pgfplotsset{compat=newest}
\pgfplotsset{every axis legend/.append style={legend cell align=left, font=\footnotesize, draw=none, fill=none}}
\pgfplotsset{every axis/.append style={axis background/.style={fill=white}}}
\pgfplotsset{every tick label/.append style={font=\footnotesize}}
\pgfplotsset{every axis label/.append style={font=\footnotesize}}

\fvset{baselinestretch=0.8}
\usepgfplotslibrary{fillbetween}
\usepgfplotslibrary{groupplots}
\usepgfplotslibrary{patchplots}
\usepgfplotslibrary{statistics}
\usepgfplotslibrary{ternary}


\usepackage{mdframed}
\definecolor{shadecolor}{rgb}{1,0.8,0.3}
\newenvironment{algorithm}[1][htbp]
{\begin{mdframed}[backgroundcolor=black!5,rightline=false,leftline=false]}
{\end{mdframed}}


\newcommand{\email}[1]{\def\@email{\texttt{\MakeLowercase{\textls[10]{#1}}}}}

% \setbeamerfont{title}{series=\bfseries} % family=\sourcesanspro
\setbeamerfont{subtitle}{series=\scshape} % family=\sourcesanspro
\setbeamercolor{subtitle}{fg=black}
\setbeamerfont{author}{family=\sourcesanspro}
\setbeamercolor{author}{fg=black}
\setbeamercolor{institute}{fg=cardinal}
\setbeamercolor{email}{fg=coolgrey}

\defbeamertemplate*{title page}{customized}[1][]
{
    \usebeamerfont{title}\textls[100]{\MakeUppercase{\textbf{\inserttitle}}}\par
    \usebeamerfont{subtitle}\usebeamercolor[fg]{subtitle}\textls[100]{\textsc{\insertsubtitle}}\par\par
    \vfill
    \usebeamerfont{author}\usebeamercolor[fg]{author}\textls[100]{\textsc{\insertauthor}}\par
    \usebeamerfont{institute}{\usebeamercolor[fg]{institute}\textls[100]{\textsc{\insertinstitute}}}\par
    \bigskip
    {\usebeamercolor[fg]{email}\@email}\par
    \usebeamerfont{date}\insertdate\par
    \usebeamercolor[fg]{titlegraphic}\inserttitlegraphic
}

% Page numbers
\addtobeamertemplate{navigation symbols}{}{%
    \usebeamerfont{footline}%
    \usebeamercolor[fg]{footline}%
    \hspace{1em}%
    \insertframenumber/\inserttotalframenumber
}
% Small overbrace
\makeatletter
\def\smalloverbrace#1{\mathop{\vbox{\m@th\ialign{##\crcr\noalign{\kern3\p@}%
  \tiny\downbracefill\crcr\noalign{\kern3\p@\nointerlineskip}%
  $\hfil\displaystyle{#1}\hfil$\crcr}}}\limits}
\makeatother

% Small underbrace
\makeatletter
\def\smallunderbrace#1{\mathop{\vtop{\m@th\ialign{##\crcr
   $\hfil\displaystyle{#1}\hfil$\crcr
   \noalign{\kern3\p@\nointerlineskip}%
   \tiny\upbracefill\crcr\noalign{\kern3\p@}}}}\limits}
\makeatother

% Over and under arrows
\newcommand{\overarrow}[2]{\overset{\mathclap{\substack{#2 \\ \downarrow}}}{#1}}
\newcommand{\underarrow}[2]{\underset{\mathclap{\substack{\uparrow \\ #2}}}{#1}}


\begin{jlcode}
	include("../../jl/support_code.jl")

	using Colors
	using ColorSchemes
	pasteljet = ColorMaps.RGBArrayMap(ColorSchemes.viridis, interpolation_levels=500, invert=true);
    pastelRedBlue = ColorMaps.RGBArrayMap([RGB(246/255, 21/255, 92/255),
										   RGB(1.0,1.0,1.0),
										   RGB( 27/255,161/255,234/255)], interpolation_levels=500);
\end{jlcode}


\setbeamercolor{itemize item}{fg=black}
\setbeamercolor{itemize subitem}{fg=black}
\setbeamercolor{itemize subsubitem}{fg=black}

\setbeamertemplate{itemize item}[circle]
\setbeamertemplate{itemize subitem}{{\textendash}}
\setbeamertemplate{itemize subsubitem}[triangle]

\setbeamerfont{frametitle}{series=\scshape} % \itshape
\setbeamercolor{frametitle}{fg=black}

\setbeamertemplate{frametitle}
{
    \vspace*{0.7cm}
    \insertframetitle
}


%%%%%%%%%%%%%%%%%%%%%%%%%%%%%%%%%%%%%%%%%%%%%%%%%%
% Math
\newcommand{\dset}{\mathcal{D}}
\newcommand{\params}{\vect \theta}

\newcommand{\true}{\text{true}}
\newcommand{\false}{\text{false}}
\newcommand{\transpose}{\top}

\newcommand{\noisy}[1]{\tilde{#1}}

\newcommand{\mat}[1]{\vect{#1}}
\renewcommand{\vec}[1]{\vect{#1}}

\usepackage{mathtools}
\DeclarePairedDelimiter{\paren}{\lparen}{\rparen}
\DeclarePairedDelimiter{\brock}{\lbrack}{\rbrack}
\DeclarePairedDelimiter{\curly}{\{}{\}}
\DeclarePairedDelimiter{\norm}{\lVert}{\rVert}
\DeclarePairedDelimiter{\abs}{\lvert}{\rvert}
\DeclarePairedDelimiter{\anglebrackets}{\langle}{\rangle}
\DeclarePairedDelimiter{\ceil}{\lceil}{\rceil}
\DeclarePairedDelimiter{\floor}{\lfloor}{\rfloor}
\DeclarePairedDelimiter{\card}{|}{|}

\newcommand{\minimize}{\operatornamewithlimits{minimize}}
\newcommand{\maximize}{\operatornamewithlimits{maximize}}
\newcommand{\supremum}{\operatornamewithlimits{supremum}}
\newcommand{\argmin}{\operatornamewithlimits{arg\,min}}
\newcommand{\argmax}{\operatornamewithlimits{arg\,max}}
\newcommand{\subjectto}{\operatorname{subject~to}}
\newcommand{\for}{\text{for} \;}
\newcommand{\dimension}[1]{\text{dim}\paren*{#1}}
\newcommand{\gaussian}[2]{\mathcal{N}(#1, #2)}
\newcommand{\Gaussian}[2]{\mathcal{N}\paren*{#1, #2}}
\newcommand{\R}{\mathbb{R}}
\newcommand{\Z}{\mathbb{Z}}
\newcommand{\N}{\mathbb{N}}
\DeclareMathOperator{\sign}{sign}
\DeclareMathOperator{\Real}{\text{Re}}
\DeclareMathOperator{\Imag}{\text{Im}}
\DeclareMathOperator{\nil}{\textsc{nil}}
\DeclareMathOperator{\Expectation}{\mathbb{E}}
\DeclareMathOperator{\Variance}{\mathrm{Var}}
\DeclareMathOperator{\Normal}{\mathcal{N}}
\DeclareMathOperator{\Uniform}{\mathcal{U}}
\DeclareMathOperator{\Dirichlet}{Dir}
\DeclareMathOperator{\atantwo}{atan2}
\DeclareMathOperator{\modOne}{mod_1}
\DeclareMathOperator{\trace}{Tr}
\newcommand{\minprob}[3]{
\begin{aligned}
	\minimize_{#1} & & #2\\
	\subjectto & & #3 \\
\end{aligned}
}